%%%%%%%%%%%%%%%%%%%%%%%%%%%%%%%%%%%%%%%%%
% Plasmati Graduate CV
% LaTeX Template
% Version 1.0 (24/3/13)
%
% This template has been downloaded from:
% http://www.LaTeXTemplates.com
%
% Author: Robert Stein (via Federica Bradascio)
%%%%%%%%%%%%%%%%%%%%%%%%%%%%%%%%%%%%%%%%%

%----------------------------------------------------------------------------------------
%	PACKAGES AND OTHER DOCUMENT CONFIGURATIONS
%----------------------------------------------------------------------------------------

\documentclass[a4paper,10pt]{article} % Default font size and paper size

%\usepackage{fontspec} % For loading fonts
%\defaultfontfeatures{Mapping=tex-text}
%\setmainfont[SmallCapsFont = Fontin SmallCaps]{Fontin} % Main document font

\usepackage{xunicode,xltxtra,url,parskip} % Formatting packages

\usepackage[usenames,dvipsnames]{xcolor} % Required for specifying custom colors

\usepackage[big]{layaureo} % Margin formatting of the A4 page, an alternative to layaureo can be \usepackage{fullpage}
% To reduce the height of the top margin uncomment: \addtolength{\voffset}{-1.3cm}

\usepackage{hyperref} % Required for adding links	and customizing them
\definecolor{linkcolour}{rgb}{0,0.2,0.6} % Link color
\hypersetup{colorlinks,breaklinks,urlcolor=linkcolour,citecolor=linkcolour,linkcolor=linkcolour} % Set link colors throughout the document
%\hypersetup{citecolor=blue, filecolor=blue,linkcolor=blue, urlcolor=blue}

%multi-row
\usepackage{multirow}
\usepackage{booktabs}
\usepackage{longtable}

\usepackage{titlesec} % Used to customize the \section command
\titleformat{\section}{\Large\scshape\raggedright}{}{0em}{}[\titlerule] % Text formatting of sections
\titlespacing{\section}{0pt}{3pt}{3pt} % Spacing around sections


% Pi� di pagina fisso
\usepackage{lastpage}
%\cfoot{\thepage\ di \pageref{LastPage}}

\usepackage{fancyhdr}
\pagestyle{fancy}
\lhead{}                %intestazione sinistra
\chead{}               %intestazione centrale
\rhead{} 			%intestazione destra
\lfoot{Publications \ of \textsc{Robert Stein}} %pi� di pagina sinistro
\cfoot{} %pi� di pagina centrale 
%\rfoot{\\thepage~/ \pageref{LastPage}} %pi� di pagina destro
\renewcommand{\headrulewidth}{0pt} %linea separazione intestazione
\renewcommand{\footrulewidth}{0.5pt} %linea separazione pi� di pagina
\fancyfoot[LE,RO]{\thepage~/ \pageref{LastPage}}%

\usepackage{array}
\newcolumntype{L}[1]{>{\raggedright\let\newline\\\arraybackslash\hspace{0pt}}m{#1}}
\newcolumntype{C}[1]{>{\centering\let\newline\\\arraybackslash\hspace{0pt}}m{#1}}
\newcolumntype{R}[1]{>{\raggedleft\let\newline\\\arraybackslash\hspace{0pt}}m{#1}}

\renewcommand{\refname}{Publications}% changes default name Bibliography to References


\begin{document}

%\pagestyle{empty} % Removes page numbering

%\parpic[r]{\includegraphics[width=0.27\textwidth]{FB_picture.pdf}}
\font\fb=''[cmr10]'' % Change the font of the \LaTeX command under the skills section
\section{Selected Peer-Reviewed Publications}
\begin{longtable}{L{10mm} p{126mm}}
	
	\bf 2020 & \textsc{M. M.~Kasliwal} et al.,
	\emph{\href{https://arxiv.org/abs/2006.11306}{Kilonova Luminosity Function Constraints based on Zwicky Transient Facility searches for 13 Neutron Star Mergers}}
	(accepted)\\
	&\emph{Developed one of three analysis pipelines used for realtime data analysis in GW follow-up. Lead or contributed to many of the individual GW follow-up campaigns, performed in real-time. Contributed to the post-search data analysis, including estimates of coverage and the statistical interpretation of non-detections.}\\
	&\\	
	
	& \textsc{R.~Stein} et al.,
	\emph{\href{https://arxiv.org/abs/2005.05340}{A high-energy neutrino coincident with a tidal disruption event}}
	(accepted)\\
	&\emph{Developed the neutrino follow-up program, led many of the neutrino campaigns. Created the program analysis pipeline, led additional post-search follow-up. Identified the Tidal Disruption Event AT2019dsg as a likely neutrino source, led the multi-wavelength modelling, statistical analysis and neutrino data analysis.}\\
	&\\	
	
	& \textsc{V.~Paliya} et al.,
	\emph{\href{https://arxiv.org/abs/2003.06012}{Multi-Frequency Observations of the Candidate Neutrino Emitting Blazar BZB J0955+3551}} 2020, ApJ, 902, 29\\
	&\emph{Performed the statistical analysis of chance coincidence,  led neutrino data analysis.}\\
	&\\	
	
	& \textsc{A.~Franckowiak} et al.,
	\emph{\href{https://arxiv.org/abs/2001.10232}{Patterns in the multi-wavelength behavior of candidate neutrino blazars}}
	2020, ApJ, 893, 162\\
	&\emph{Contributed to the discussion of statistical analysis and interpretation.}\\
	&\\	
	
	& \textsc{S. van Velzen} et al.,
	\emph{\href{https://arxiv.org/abs/2001.01409}{Seventeen Tidal Disruption Events from the First Half of ZTF Survey Observations: Entering a New Era of Population Studies}}
	(accepted)\\
	&\emph{Technical implementation of filtering and analysis pipeline, code development}\\
	&\\	
\end{longtable}

\section{Selected Proceedings and Publications in Prep.}

\begin{longtable}{L{10mm} p{126mm}}

		\bf 2020 & \textsc{T. Ahumada et al.},
		\emph{Ab Whiskey: Identification of the Afterglow of the Short-Duration Gamma-Ray Burst GRB 200826A with the Zwicky Transient Facility},
		(in prep.)\\
		&\emph{Created analysis pipeline. Participated in all individual GRB follow-up campaigns, performed real-time data analysis to identify candidate afterglows.}\\
		&\\	
		
		& \textsc{S. Reusch et al.},
		\emph{Observations of bright nuclear transient AT2019fdr coincident with high-energy neutrino IceCube-200530A},
		(in prep.)\\
		&\emph{My analysis pipeline was used for real-time data analysis, participated in all neutrino follow-up campaigns. Led the both statistical and neutrino data analysis of this work, and contributed radio data as PI of a successful VLA DDT proposal.}\\
		&\\	
		%	& \textsc{IceCube Collaboration},
		%	\emph{Search for high-energy neutrinos from tidal disruption events},
		%	(in prep.)\\
		%	&\\	
	
		\bf 2019 & \textsc{R.~Stein for the IceCube Collaboration},
	\emph{\href{https://arxiv.org/abs/1908.08547}{Search for Neutrinos from Populations of Optical Transients}},
	PoS(ICRC2019)1016\\
	&\emph{Developed likelihood analysis code, used by the IceCube collaboration for neutrino astronomy searches. Used this code to performed TDE-neutrino correlation study. Compiled a catalogue of TDEs based on published examples in the literature. Developed code to perform additional cosmology calculations for deriving the diffuse flux associated with a neutrino source population, and used this to set limits on the contribution of TDEs to the neutrino flux.}\\
	&\\
\end{longtable}

\end{document}