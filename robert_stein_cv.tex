%%%%%%%%%%%%%%%%%%%%%%%%%%%%%%%%%%%%%%%%%
% Plasmati Graduate CV
% LaTeX Template
% Version 1.0 (24/3/13)
%
% This template has been downloaded from:
% http://www.LaTeXTemplates.com
%
% Author: Robert Stein (via Federica Bradascio)
%%%%%%%%%%%%%%%%%%%%%%%%%%%%%%%%%%%%%%%%%

%----------------------------------------------------------------------------------------
%	PACKAGES AND OTHER DOCUMENT CONFIGURATIONS
%----------------------------------------------------------------------------------------

\documentclass[a4paper,10pt]{article} % Default font size and paper size

%\usepackage{fontspec} % For loading fonts
%\defaultfontfeatures{Mapping=tex-text}
%\setmainfont[SmallCapsFont = Fontin SmallCaps]{Fontin} % Main document font

\usepackage{xunicode,xltxtra,url,parskip} % Formatting packages

\usepackage[usenames,dvipsnames]{xcolor} % Required for specifying custom colors

\usepackage[big]{layaureo} % Margin formatting of the A4 page, an alternative to layaureo can be \usepackage{fullpage}
% To reduce the height of the top margin uncomment: \addtolength{\voffset}{-1.3cm}

\usepackage{hyperref} % Required for adding links	and customizing them
\definecolor{linkcolour}{rgb}{0,0.2,0.6} % Link color
\hypersetup{colorlinks,breaklinks,urlcolor=linkcolour,citecolor=linkcolour,linkcolor=linkcolour} % Set link colors throughout the document
%\hypersetup{citecolor=blue, filecolor=blue,linkcolor=blue, urlcolor=blue}

%multi-row
\usepackage{multirow}
\usepackage{booktabs}
\usepackage{longtable}

\usepackage{titlesec} % Used to customize the \section command
\titleformat{\section}{\Large\scshape\raggedright}{}{0em}{}[\titlerule] % Text formatting of sections
\titlespacing{\section}{0pt}{3pt}{3pt} % Spacing around sections


% Pi� di pagina fisso
\usepackage{lastpage}
%\cfoot{\thepage\ di \pageref{LastPage}}

\usepackage{fancyhdr}
\pagestyle{fancy}
\lhead{}                %intestazione sinistra
\chead{}               %intestazione centrale
\rhead{} 			%intestazione destra
\lfoot{Curriculum Vit\ae \ of \textsc{Robert Stein}} %pi� di pagina sinistro
\cfoot{} %pi� di pagina centrale 
%\rfoot{\\thepage~/ \pageref{LastPage}} %pi� di pagina destro
\renewcommand{\headrulewidth}{0pt} %linea separazione intestazione
\renewcommand{\footrulewidth}{0.5pt} %linea separazione pi� di pagina
\fancyfoot[LE,RO]{\thepage~/ \pageref{LastPage}}%

\usepackage{array}
\newcolumntype{L}[1]{>{\raggedright\let\newline\\\arraybackslash\hspace{0pt}}m{#1}}
\newcolumntype{C}[1]{>{\centering\let\newline\\\arraybackslash\hspace{0pt}}m{#1}}
\newcolumntype{R}[1]{>{\raggedleft\let\newline\\\arraybackslash\hspace{0pt}}m{#1}}

\renewcommand{\refname}{Publications}% changes default name Bibliography to References


\begin{document}

%\pagestyle{empty} % Removes page numbering

%\parpic[r]{\includegraphics[width=0.27\textwidth]{FB_picture.pdf}}
\font\fb=''[cmr10]'' % Change the font of the \LaTeX command under the skills section



%----------------------------------------------------------------------------------------
%	NAME AND CONTACT INFORMATION
%----------------------------------------------------------------------------------------
%\par{\centering \includegraphics[width=0.25\textwidth]{FB_picture4.png} \qquad {\Huge Federica \textsc{Bradascio}}\bigskip\par}
\par{\centering \qquad {\Huge Robert \textsc{Stein}}\bigskip\par}

\section{Personal Data}

\begin{tabular}{rl}
\textsc{Place and Date of Birth:} & London  | 10 June 1995 \\
%\textsc{Address:} & Eichbuschallee 51, Berlin, Germany\\
\textsc{Nationality:} & British \& Irish\\
\textsc{Email:} & \href{mailto:robert.stein@desy.de}{robert.stein@desy.de} \\
\textsc{Website:} & \href{https://robertdstein.github.io}{robertdstein.github.io} \\
%\textsc{Orcid:} & \href{https://orcid.org/0000-0003-2434-0387}{0000-0003-2434-0387} \\
\end{tabular}
%----------------------------------------------------------------------------------------
%	RESEARCH
%----------------------------------------------------------------------------------------
%\section{Research}
%\begin{longtable}{r| p{108mm}}	
%
%\textsc{1 July. 2017 -- present}& {\parbox{110mm}{\rule{0pt}{2pt}PhD student in \href{https://www-zeuthen.desy.de/~afrancko/}{Multi-Messenger Group}, \textbf{DESY Zeuthen}}}\\
%& {\parbox{108mm}{\rule{0pt}{2pt}\small{\begin{itemize}
%				\item Member of the IceCube collaboration. Developed likelihood analysis code \emph{\href{https://github.com/IceCubeOpenSource/flarestack}{Flarestack}}, used by collaboration members for several cross-correlation analyses. Performed the first cross-correlation analysis of \href{https://arxiv.org/abs/1908.08547}{neutrinos and tidal disruption events}. Served as the realtime shifter for two years, and led the response to many IceCube neutrino alerts. 
%				\item Contributor to \emph{\href{https://arxiv.org/abs/1904.05922}{Ampel}} platform. Developed the \emph{\href{https://github.com/robertdstein/ampel_followup_pipeline}{Ampel Follow-up Pipeline}} to identify objects coincident with external triggers, used for ZTF multi-messenger follow-up campaigns. 
%				\item Member of the the ZTF collaboration. Led many ZTF neutrino follow-up campaigns. Identified the tidal disruption event \href{https://arxiv.org/abs/2005.05340}{AT2019dsg} as a candidate neutrino source. Led or participated in many ZTF searches for electromagnetic counterparts to gravitational waves and gamma-ray bursts.
%			\end{itemize}}}}\\
%\multicolumn{2}{c}{} \\
%
%\textsc{Nov. 2015 -- July 2016} & \parbox{108mm}{\rule{0pt}{2pt}Master student in \href{http://www.iexp.uni-hamburg.de/groups/astroparticle/en/}{Astroparticle Group}, \textbf{University of Hamburg}}\\
%&{\parbox{108mm}{\rule{0pt}{2pt}\small{
%			\begin{itemize}
%				\item Simulated cosmic ray air showers using CORSIKA, using HESS detector as a baseline.
%				Trained a BDT to identify pixels in IACT camera images that are dominated by direct Cherenkov light from primary particles. 
%				\item Developed a generic method to reconstruct cosmic ray events using the distribution of measured direct Cherenkov emission across cameras. Method reached cosmic ray charge resolution of $\sim$1 on simulated data, which would be sufficient for composition studies. Estimated performance expected for various CTA geometries.
%			\end{itemize}}}}\\
%\multicolumn{2}{c}{} \\
%\end{longtable}

%\bigskip
%----------------------------------------------------------------------------------------
%	EDUCATION
%----------------------------------------------------------------------------------------

\section{Education}
%\section{Academic Career}

\begin{longtable}{R{26mm}|p{110mm}}

%\textsc{1 Sep. 2016 - today}& {\parbox{110mm}{\rule{0pt}{2pt}Member of the IceCube Collaboration}}\\
%\multicolumn{2}{c}{} \\

\textsc{July 2017 --}& {\parbox{108mm}{\rule{0pt}{2pt}PhD in \textsc{}\textsc{Experimental Physics},}}\\
\textsc{Dec. 2020}& \normalsize\textbf{Humboldt University of Berlin / DESY Zeuthen}\\
& Thesis: ``\emph{Search for multi-messenger sources of neutrinos and gravitational waves}" (in prep.)\\
&\small{Research Advisor:} \normalsize \textsc{A. Franckowiack}\\
& \begin{itemize}
	\item Cross-correlation of neutrinos with multi-wavelength catalogues
	\item Led response to neutrino alerts as the \emph{IceCube realtime shifter}
	\item ZTF follow-up of neutrino/gravitational wave/GRB events
%	\item Participated in ZTF GW/GRB follow-up program.
\end{itemize}\\
\multicolumn{2}{c}{} \\

\textsc{Sep. 2013 --}& {\parbox{108mm}{\rule{0pt}{2pt}MSci in \textsc{}\textsc{Physics with a Year In Europe},}}\\
\textsc{June 2017}& \normalsize\textbf{Imperial College London / University of Hamburg}\\
& Thesis: ``\emph{\href{https://robertdstein.github.io/documents/robert_stein_master_thesis.pdf}{Reconstruction of Charge Number of Heavy Cosmic Rays using Cherenkov Light}}"\\
&\small{Research Advisor:} \normalsize \textsc{D. Horns} (University of Hamburg)\\
& {\parbox{108mm}{\rule{0pt}{2pt}\normalsize{Graduated with First Class Honours}}}\\
& \begin{itemize}
	\item Development of novel reconstruction method for heavy cosmic rays detected by IACTs, using direct Cherenkov light
	\item Estimates of performance for simulated CTA geometries
\end{itemize}\\
\multicolumn{2}{c}{} \\
%------------------------------------------------
\end{longtable}

%----------------------------------------------------------------------------------------
%	PRESENTATIONS
%----------------------------------------------------------------------------------------

\section{Selected Talks}

\begin{longtable}{R{26mm}|p{110mm}}
	
	\textsc{14\textsuperscript{th} Oct. 2020} & {\textsc{Invited Talk}, ASTRON Astrolunch, \normalsize{Dwingeloo, NL}}\\
	& ``\emph{A high-energy neutrino coincident with a tidal disruption event}''  \\
	\multicolumn{2}{c}{} \\
	
	\textsc{25\textsuperscript{th} Aug. 2020} & {\textsc{Invited Talk}, NASA GSFC ASD Colloquium, \normalsize{Greenbelt, USA}}\\
	& ``\emph{A high-energy neutrino coincident with a tidal disruption event}''  \\
	\multicolumn{2}{c}{} \\
	
	%\textsc{22 June - 2 July 2020} & {\textsc{Poster}, 29\textsuperscript{th} International Conference on Neutrino Physics (Neutrino), \normalsize{Chicago, USA}}\\
	%& ``\emph{\href{https://robertdstein.github.io/documents/neutrino_2020_347_stein.pdf}{A high-energy neutrino coincident with a tidal disruption event}}''  \\
	%\multicolumn{2}{c}{} \\
	
	\textsc{5\textsuperscript{th} June 2020} & {\textsc{Invited Talk}, DESY Astroparticle Seminar, \normalsize{Zeuthen, DE}}\\
	& ``\emph{A high-energy neutrino coincident with a tidal disruption event}''  \\
	\multicolumn{2}{c}{} \\
	
	%\textsc{17 Dec. 2019} & {\textsc{Talk}, TEXAS Symposium for Relativistic Astrophysics, \normalsize{Portsmouth, UK}}\\
	%& ``\emph{The IceCube Realtime System}''  \\
	%\multicolumn{2}{c}{} \\
	
	\textsc{26\textsuperscript{th} Oct. 2019} & {\textsc{Invited Talk}, PAHEN Conference, \normalsize{Berlin, DE}}\\
	& ``\emph{Neutrinos from optical transients with IceCube}''  \\
	\multicolumn{2}{c}{} \\
	
	%\textsc{24 July-1 Aug. 2019} & {{\textsc{Talk}, 36\textsuperscript{th} International Cosmic Ray Conference (ICRC)}, \normalsize{Madison, USA}}\\
	%& ``\emph{Search for Neutrinos from Populations of Optical Transients}''  \\
	%\multicolumn{2}{c}{} \\
	
	%\textsc{25 - 29 Mar. 2019} & {\textsc{Talk}, The New Era of Multi-Messenger Astrophysics, \normalsize{Groningen, NL}}\\
	%& ``\emph{Search for High-Energy Neutrinos from Populations of Optical Transients}''  \\
	%\multicolumn{2}{c}{} \\
	
	%\textsc{27-31 Aug. 2018 } & {\textsc{Poster}, TeV Particle Astrophysics (TeVPA), \normalsize{Berlin, DE}}\\
	%& ``\emph{\href{https://robertdstein.github.io/documents/robert_stein_TeVPA_2018_poster.pdf}{Probing the Tidal Disruption of Stars by Supermassive Black Holes with the IceCube Neutrino Observatory}}''  \\
	%\multicolumn{2}{c}{} \\
	
	\textsc{30\textsuperscript{th} July 2018} & {\textsc{Invited Talk}, ESO Thirty Minute Talk, \normalsize{Santiago, CL}}\\
	& ``\emph{ZTF and the AMPEL Broker: Providing a realtime public astronomy survey}''  \\
	\multicolumn{2}{c}{} \\
	
	%\textsc{20-26 Jan. 2018 } & {\textsc{Talk}, Using Tidal Disruption Events to Study Super-Massive Black Holes, \normalsize{Aspen, USA}}\\
	%& ``\emph{The search for high-energy neutrinos: a TDE stacking analysis}'' \\
	%\multicolumn{2}{c}{} \\
	
\end{longtable}
\newpage

\section{Scholarships, Awards and Honours}

\begin{longtable}{R{26mm}|p{110mm}}
	\textsc{2\textsuperscript{nd} July 2020} & Winner of first session poster competition, Neutrino 2020 Conference \\
	\multicolumn{2}{c}{} \\
	\textsc{16\textsuperscript{th} Oct 2019} & Winner of the annual DESY Science Slam, DESY Hamburg\\
	\multicolumn{2}{c}{} \\
	\textsc{21\textsuperscript{st} Nov 2018} & Winner of the annual Zeuthen Science Slam, DESY Zeuthen\\
	\multicolumn{2}{c}{} \\
	%	\textsc{June - Sep. 2015}  & DAAD RISE Scholarship, TU Dortmund\\
	%	\multicolumn{2}{c}{} \\
	%	\textsc{Oct. 2013 - June 2017} & President's Undergraduate Scholarship, Imperial College London\\
	%	\multicolumn{2}{c}{} \\
\end{longtable}

\section{Selected Telescope Time Awarded}
%\begin{longtable}{r | p{108mm}}	
\begin{longtable}{R{28mm}|p{108mm}}
\textsc{Oct. 2020 --}& {\parbox{108mm}{\rule{0pt}{2pt}Australia Telescope Compact Array Program (Co-I)}}\\
\textsc{Mar. 2021}&  \normalsize\emph{Radio emission from stellar tidal disruption flares}\\
\multicolumn{2}{c}{} \\
\textsc{Sep. 2020 --}& {\parbox{108mm}{\rule{0pt}{2pt}Gran Telescopio Canarias Program (Co-I)}}\\
\textsc{Feb. 2021 }&  \normalsize\emph{Spectroscopic classification of potential neutrino counterparts identified by ZTF}\\
\multicolumn{2}{c}{} \\
\textsc{June 2020 --}& {\parbox{108mm}{\rule{0pt}{2pt}Very Large Array Program (PI)}}\\
\textsc{Present}&  \normalsize\emph{VLA observations to establish the neutrino counterpart to a giant AGN flare}\\
\multicolumn{2}{c}{} \\
\end{longtable}

%----------------------------------------------------------------------------------------
%	TEACHING
%----------------------------------------------------------------------------------------
\section{Supervision, Teaching and Outreach}
%\begin{longtable}{r | p{108mm}}	
\begin{longtable}{R{28mm}|p{108mm}}
	\textsc{Oct. 2019 --}& {\parbox{108mm}{\rule{0pt}{2pt} Supervision of master's degree student:  \textsc{J. Necker}}}\\
	\textsc{Oct. 2020} &  \normalsize\emph{Search for high-energy neutrinos from core-collapse supernovae}\\
	\multicolumn{2}{c}{} \\
	\textsc{Sep. 2019 -- }& {\parbox{108mm}{\rule{0pt}{2pt} Supervision of master's degree student:  \textsc{R. Naab}}}\\
	\textsc{Sep. 2020}&  \normalsize\emph{The next-generation Optical Follow-Up (OFU) program for IceCube}\\
	\multicolumn{2}{c}{} \\
	\textsc{Oct 2018 -- }& {\parbox{108mm}{\rule{0pt}{2pt} Supervision of bachelor's degree student:  \textsc{A. Vagts}}}\\
	\textsc{Aug. 2019} &  \normalsize\emph{Investigation of the TXS 0506+056 neutrino spectrum}\\
	\multicolumn{2}{c}{} \\
	\hline
	\multicolumn{2}{c}{} \\
	\textsc{June 2018 --  July 2019}& {\parbox{108mm}{\rule{0pt}{2pt}Teaching Assistant: \emph{Experimental Astroparticle Physics} (2 semesters)}}\\
	\multicolumn{2}{c}{} \\
	\hline
	\multicolumn{2}{c}{} \\
	\textsc{Oct. 2018 -- Nov. 2019}& {\parbox{108mm}{\rule{0pt}{2pt} Volunteer: \emph{International Cosmic Day} (2 years)}}\\
	\multicolumn{2}{c}{} \\
	\textsc{June 2018 -- June 2019}& {\parbox{108mm}{\rule{0pt}{2pt} Volunteer: \emph{Lange Nacht der Wissenschaft} (2 years)}}\\
	\multicolumn{2}{c}{} \\
	\textsc{March 2018 }& {\parbox{108mm}{\rule{0pt}{2pt} Organiser: \emph{IceCube Masterclass}}}\\
	\multicolumn{2}{c}{} \\
\end{longtable}

%----------------------------------------------------------------------------------------
%	INTERNSHIPS AND SUMMER SCHOOLS
%----------------------------------------------------------------------------------------

%\section{Internships and Summer Schools}
%\section{Research Internships and Physics Summer Schools}
%\section{Research Internships and Physics Schools}
%
%\begin{longtable}{R{28mm}|p{108mm}}
%
%\textsc{3 -- 11 Oct. 2018}& \textsc{School for Astroparticle Physics}, \normalsize\textbf{Obertrubach-B\"arnfels, Germany}\\
%\multicolumn{2}{c}{} \\
%
%\textsc{June -- Sep. 2015} & \textsc{DAAD RISE Summer Student Program}, \normalsize\textbf{T.U. Dortmund, DE}\\
%&{\parbox{108mm}{\rule{0pt}{2pt}\small{Worked on the LHCb Data Analysis, using Python/PyROOT. \\}}}\\
%&\small{Supervisor:} \normalsize \textsc{T. Tekampe}\\
%\multicolumn{2}{c}{} \\
%\end{longtable}
%
%\bigskip


%%----------------------------------------------------------------------------------------
%%	CONFERENCES AND WORKSHOP
%%----------------------------------------------------------------------------------------
%
%\section{Conferences and Workshops Attended}
%
%\begin{longtable}{R{26mm}|p{110mm}}
%\textsc{24 July-1 Aug. 2019} & {36\textsuperscript{th} International Cosmic Ray Conference (ICRC)}, \normalsize{Madison}\\
%\multicolumn{2}{c}{} \\
%
%\textsc{2-4 Oct. 2018} & {Very Large Volume Neutrino Telescopes (VLVNT) 2018 Workshop}, \normalsize{Dubna}\\
%\multicolumn{2}{c}{} \\
%
%\textsc{7-8 Oct. 2017} & {MANTS 2017 Meeting}, \normalsize{Marseille}\\
%\multicolumn{2}{c}{} \\
%
%\textsc{3-10 Aug. 2016} & {38\textsuperscript{th} International Conference on High Energy Physics} (ICHEP), \normalsize{Chicago}\\
%%& Poster presentation and elevator talk in the session ``1' elevator speeches by young scientists'' on: ``\emph{Studies of the impact of magnetic field uncertainties on the physics parameters of the Mu2e experiment}''  \\
%\multicolumn{2}{c}{} \\
%
%\textsc{15-16 June 2016} & {49\textsuperscript{th} Annual Fermilab Users Meeting}, \normalsize{FERMILAB}\\
%%& Poster presented, 2\textsuperscript{nd} award won on:  ``\emph{Studies of the Impact of Field Uncertainties on Physics Parameters of the Mu2e Experiment}''\\
%\multicolumn{2}{c}{} \\
%
%\textsc{4-6 Dec. 2015}& {Dark Matter at a future hadron collider}, \normalsize{FERMILAB}\\
%\multicolumn{2}{c}{} \\
%
%\end{longtable}
%\bigskip


%----------------------------------------------------------------------------------------
%	PUBLICATIONS
%----------------------------------------------------------------------------------------

%\begin{thebibliography}{10}% Bibliography - this is intentionally simple in this template
%\bibitem{Bradascio:2016hjx}
%\textsc{F.~Bradascio},
%\emph{Studies of the Impact of Magnetic Field Uncertainties on Physics Parameters of the Mu2e Experiment},
%FERMILAB-MASTERS-2016-04.
%
%\bibitem{Bradascio:PoS}
%\textsc{F.~Bradascio},
%\emph{Studies of the Impact of Magnetic Field Uncertainties on Physics Parameters of the Mu2e Experiment},
%to appear in  the proceedings of the ICHEP conference.
%
%\bibitem{}
%\textsc{M. Lopes, et al.},
%\emph{Mu2e Transport Solenoid Cold-Mass Alignment Issues},
%ASC2016-2LOr1C-04.
%\end{thebibliography}

%\bibliographystyle{apsrev}
%%\bibliographystyle{ieeetr}
%\bibliography{CVbib}{}


%\bigskip

%\bigskip

%----------------------------------------------------------------------------------------
%	AWARDS AND HONORS
%----------------------------------------------------------------------------------------

%\bigskip

%----------------------------------------------------------------------------------------
%	OUTREACH
%----------------------------------------------------------------------------------------

%\section{Outreach}
%
%\begin{longtable}{R{26mm}|p{110mm}}
%\textsc{Nov. 2019} & \textsc{International Cosmic Day 2019}, DESY Zeuthen\\
%\multicolumn{2}{c}{} \\
%
%\textsc{Nov. 2018} & \textsc{International Cosmic Day 2018}, DESY Zeuthen\\
%\multicolumn{2}{c}{} \\
%
%\textsc{March 2018} & \textsc{IceCube Masterclass}, DESY Zeuthen\\
%\multicolumn{2}{c}{} \\
%
%\textsc{Nov. 2017} & \textsc{International Cosmic Day 2017}, DESY Zeuthen\\
%\multicolumn{2}{c}{} \\
%
%\textsc{Oct. 2017} & \textsc{IceCube Masterclass}, DESY {\tiny }Zeuthen\\
%\multicolumn{2}{c}{} \\
%\end{longtable}
%
%%\bigskip


%%----------------------------------------------------------------------------------------
%%	LINGUISTIC TRAINING SUMMER SCHOOLS
%%----------------------------------------------------------------------------------------
%
%\section{Linguistic Training Summer Schools}
%\begin{longtable}{R{26mm}|p{110mm}}
%\textsc{July 2008}& \textsc{English Language Course}, \normalsize\textbf{Hampstead School of English}, London\\
%&{\parbox{110mm}{\rule{0pt}{2pt}\small{Two weeks English language course at \emph{Proficient User (C1) level}}}}\\
%\multicolumn{2}{c}{} \\
%
%\textsc{July 2007}& \parbox{110mm}{\rule{0pt}{2pt}\textsc{English Language Course}, \normalsize\textbf{Twin English Center London}, Stanford (UK)}\\
%&{\parbox{110mm}{\rule{0pt}{2pt}\small{Two weeks English language course at \emph{Upper Intermediate level}}}}\\
%& \small{Grade:} \normalsize{A} \\
%\multicolumn{2}{c}{} \\
%
%\textsc{August 2006}& \textsc{English Language Course}, \normalsize\textbf{Lewis School of English}, Perth (UK)\\
%&{\parbox{110mm}{\rule{0pt}{2pt}\small{Two weeks English language course at \emph{Intermediate level}}}}\\
%& {\parbox{110mm}{\rule{0pt}{2pt}\small{Final score:} \\
%\small{\textsc{Reading:} Higher Intermediate;\\ 
%\textsc{Writing:} Higher Intermediate;\\ 
%\textsc{Listening:} Plus Intermediate+;\\ 
%\textsc{Speaking:} Plus Intermediate+. }}}\\
%\multicolumn{2}{c}{} \\
%\end{longtable}
%
%%\newpage
%
%%\bigskip



%----------------------------------------------------------------------------------------
%	COMPUTER SKILLS 
%----------------------------------------------------------------------------------------

\section{Additional Information}

\begin{longtable}{rl}
%\textbf{Advanced Knowledge:} & Bash, Shell, C/C++, Python, {\fb \LaTeX}, Inkscape \setmainfont[SmallCapsFont=Fontin SmallCaps]{Fontin-Regular}\\
%&\\
%\textbf{Intermediate Knowledge:} & {\parbox{95mm}{\rule{0pt}{2pt}Root Data Analysis, Matlab}}\\
%&\\
%\textbf{Basic Knowledge:} & Mathematica, Awk, MongoDB\\
%&\\

\textbf{Collaboration Membership} & IceCube, Zwicky Transient Facility (ZTF)\\
\textbf{Programming Skills:} & Python, \LaTeX , Bash (Advanced)  \\
\textbf{Language Skills:} & English (Native Speaker), German (Advanced - C1)  \\

\end{longtable}

%\begin{tabular}{lll}
%{Mother tongue:} & \textsc{Italian} &\\
%{Other languages:} & \textsc{English}&\\
%& \textbf{Listening:} & Good\\
%& \textbf{Reading:} & Excellent \\
%& \textbf{Writing:} & Good\\
%& \textbf{Speaking:} & Fluent\\
%\end{tabular}

%\bigskip

\newpage

\section{Selected Publications}
\begin{longtable}{L{10mm} p{126mm}}
	\bf 2020 & \textsc{S. Reusch et al.},
	\emph{Observations of bright nuclear transient AT2019fdr coincident with high-energy neutrino IceCube-200530A},
	(in prep.)\\
	&\emph{Realtime follow-up and data analysis, statistical analysis, contributed radio data}\\
	&\\	
%	& \textsc{IceCube Collaboration},
%	\emph{Search for high-energy neutrinos from tidal disruption events},
%	(in prep.)\\
%	&\\	
	& \textsc{M. M.~Kasliwal} et al.,
	\emph{\href{https://arxiv.org/abs/2006.11306}{Kilonova Luminosity Function Constraints based on Zwicky Transient Facility searches for 13 Neutron Star Mergers}}
	(submitted)\\
	&\emph{Developed one of three analysis frameworks, realtime follow-up and data analysis}\\
	&\\	
	
	& \textsc{R.~Stein} et al.,
	\emph{\href{https://arxiv.org/abs/2005.05340}{A high-energy neutrino coincident with a tidal disruption event}}
	(submitted)\\
	&\emph{Developed analysis framework, led follow-up program, modelling, statistical analysis}\\
	&\\	
	
	& \textsc{V.~Paliya} et al.,
	\emph{\href{https://arxiv.org/abs/2003.06012}{Multi-Frequency Observations of the Candidate Neutrino Emitting Blazar BZB J0955+3551}} (submitted)\\
	&\emph{Statistical analysis of chance coincidence probability, led neutrino data analysis}\\
	&\\	
	
	& \textsc{A.~Franckowiak} et al.,
	\emph{\href{https://arxiv.org/abs/2001.10232}{Patterns in the multi-wavelength behavior of candidate neutrino blazars}}
	2020, ApJ, 893, 162\\
	&\emph{Contributed to the discussion and interpretation of neutrino correlation}\\
	&\\	
	
	\bf 2019 & \textsc{R.~Stein for the IceCube Collaboration},
	\emph{\href{https://arxiv.org/abs/1908.08547}{Search for Neutrinos from Populations of Optical Transients}},
	PoS(ICRC2019)1016\\
	&\emph{Developed likelihood analysis code, TDE catalogue compilation, data analysis}\\
	&\\
\end{longtable}

\section{Selected Software}
\begin{longtable}{L{10mm} p{126mm}}
	
	\bf 2020 & \textsc{R.~Stein et al.},
	\emph{\href{https://github.com/robertdstein/ampel_followup_pipeline}{Ampel Follow-up Pipeline}}, DOI: \href{https://doi.org/10.5281/zenodo.4048335}{10.5281/zenodo.4048335}\\
	&\emph{Python code for ZTF data analysis, built using the AMPEL framework. Primarily used for neutrino, gravitational wave and gamma-ray burst searches.}\\
	&\\	
	
	& \textsc{R.~Stein et al.},
	\emph{\href{https://github.com/IceCubeOpenSource/flarestack}{Flarestack}}, DOI: \href{https://doi.org/10.5281/zenodo.3619383}{10.5281/zenodo.3619383}\\
	&\emph{Likelihood analysis python code for neutrino data analysis, as well as for neutrino population and cosmology calculations}\\
	&\\
\end{longtable}

\end{document}